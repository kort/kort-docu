% Neue Seite beginnen
\cleardoublepage

% Stil des Abstract-Titels veraendern
\renewcommand{\abstractname}{{\Huge\bfseries Abstract}}
% Titel auch in Kopfzeile anzeigen
\markboth{Abstract}{Abstract}

\begin{abstract}
% Kopf- und Fusszeile auch auf Abstractseite
\thispagestyle{scrheadings}
\brand{OpenStreetMap} ist ein freies Projekt, welches jedermann ermöglicht, Kartendaten zu nutzen und zu editieren.
Durch diesen öffentlichen Charakter ist es nicht ausgeschlossen, dass fehlerhafte bzw. unvollständige Daten eingetragen werden.
Es gibt verschiedene Tools, die es sich zum Ziel gesetzt haben, solche Fehler zu finden und aufzubereiten.
Die so aufbereiteten Daten mussten bisher anschliessend manuell verglichen und korrigiert werden.

Zur Behebung dieser Fehler ist die cross-platform \gls{WebApp} \kort{} entwickelt worden.
Diese ist in JavaScript geschrieben und basiert auf dem \brand{Sencha Touch 2}-Framework.
Im Backend kommt eine \brand{PostgreSQL}-Datenbank zum Einsatz. Die komplette Kommunikation ist mit \gls{REST}-Schnittstellen realisiert.

Mit \kort{} soll das Verbessern von Karten-Daten auf unterhaltsame Weise ermöglicht werden.
Dem Benutzer werden dazu die \brand{OpenStreetMap}-Fehler auf einer Karte angezeigt.
Falls er die Lösung für einen dieser Fehler kennt, kann er einen entsprechenden Lösungsvorschlag abgeben.
Weitere Spieler der App überprüfen daraufhin den Vorschlag auf seine Gültigkeit.
Um die Benutzer zu motivieren, die App regelmässig zu verwenden, wurden zahlreiche Spiel-Elemente eingesetzt.
So erhält ein Benutzer für alle Aktionen Punkte (sogenannte \emph{Koins}), die dann in einer Rangliste zum Vergleich dargestellt werden.
Zudem können Auszeichnungen für besondere Leistungen gewonnen werden.
Dieser Ansatz ist bekannt unter dem Begriff \emph{\gls{Gamification}}.

\end{abstract}