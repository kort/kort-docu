\chapter*{Inhalt der CD}
% Titel auch in Kopfzeile anzeigen
\markboth{Inhalt der CD}{Inhalt der CD}
% Kapitel in Inhaltsverzeichnis einfügen
\addcontentsline{toc}{chapter}{Inhalt der CD}

\begin{table}[H]
\centering
\begin{tabular}{|p{0.35\twocelltabwidth}|p{0.65\twocelltabwidth}|}
\hline 
\textbf{Pfad} & \textbf{Beschreibung} \\ 
\hline 
\inlinecode{.sencha/} & Konfiguration für Sencha Cmd \\ 
\hline 
\inlinecode{\_DESIGN/} & Grafik-Rohdaten \\ 
\hline 
\inlinecode{\_DOCUMENTATION/} & Dokumentation der Arbeit \\ 
\hline 
\inlinecode{\_DOCUMENTATION/ba-kort-
\newline jhunzike\_soderbol.pdf} & Dokumentation der Arbeit im PDF-Format \\ 
\hline 
\inlinecode{app/} & \kort{} Frontend \\ 
\hline 
\inlinecode{build/Kort/production/} & \kort{} Production Build (komprimierte JavaScript Sourcefiles) \\ 
\hline 
\inlinecode{docs/} & Generierte Code-Dokumentationen (\kort{} Frontend, \kort{} Backend, Ext.ux.LeafletMap) \\ 
\hline 
\inlinecode{examples/} & Frontend Prototypen \\ 
\hline 
\inlinecode{i18n/} & Internationalisierungs-Plugin für Sencha Touch \\ 
\hline 
\inlinecode{jsduck/} & Konfiguration zur Generierung der JSDuck Code-Dokumentation \\ 
\hline 
\inlinecode{lib/} & Verwendete Library-Pakete \\ 
\hline 
\inlinecode{patch/} & Sencha Touch Bugfixes \\ 
\hline 
\inlinecode{resources/} & Ressourcen, welche von \kort{} verwendet werden (CSS, Bilder, Sprach-Property-Files) \\ 
\hline 
\inlinecode{server/} & \kort{} Backend \\ 
\hline 
\inlinecode{test/} & Tests der Applikation \\ 
\hline 
\inlinecode{touch/} & Sencha Touch 2 Library \\ 
\hline 
\inlinecode{ux/} & Sencha Touch Erweiterungs-Komponenten \\ 
\hline 
\inlinecode{.travis.yml} & Konfiguration für Travis CI \\ 
\hline 
\inlinecode{app.js} & Einstiegspunkt des \kort{} Frontends \\ 
\hline 
\inlinecode{app.json} & Sencha Cmd Konfiguration von \kort{} \\ 
\hline 
\inlinecode{index.html} & Startseite \\ 
\hline 
\end{tabular}
\end{table}